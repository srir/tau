%%%%%%%%%%%%%%%%%%%%%%%%%%%%%%%%%%%%%%%%%%%%%%%%%%%%%%%%%%%%%%%%%%%%%%%%%%%%
%%%                        Beginning Instructions                        %%%
%%%%%%%%%%%%%%%%%%%%%%%%%%%%%%%%%%%%%%%%%%%%%%%%%%%%%%%%%%%%%%%%%%%%%%%%%%%%
% If you are just trying to typset your homework, and you don't want to    %
% change any settings, skip downwards to the part that                     %
% says "SKIP TO HERE..."                                                   %
%%%%%%%%%%%%%%%%%%%%%%%%%%%%%%%%%%%%%%%%%%%%%%%%%%%%%%%%%%%%%%%%%%%%%%%%%%%%

\documentclass[11pt]{article}

%%%%%%%%%%%%%%%%%%%%%%%%%%%%%%%%%%%%%%%%%%%%%%%%%%%%%%%%%%%%%%%%%%%%%%%%%%%%
%%%                            Package Imports                           %%%
%%%%%%%%%%%%%%%%%%%%%%%%%%%%%%%%%%%%%%%%%%%%%%%%%%%%%%%%%%%%%%%%%%%%%%%%%%%%
% Packages in LaTeX are similar to packages in Java. They allow you to use %
% pre-written libraries to typeset more complicated things.                %
%%%%%%%%%%%%%%%%%%%%%%%%%%%%%%%%%%%%%%%%%%%%%%%%%%%%%%%%%%%%%%%%%%%%%%%%%%%%
\usepackage{amsmath, amsfonts, amsthm, amssymb}  % Some math symbols
\usepackage{enumerate}
\usepackage{color, graphicx, float}
\usepackage{fancyhdr} % Pretty headers
\usepackage{hyperref}
%%%%%%%%%%%%%%%%%%%%%%%%%%%%%%%%%%%%%%%%%%%%%%%%%%%%%%%%%%%%%%%%%%%%%%%%%%%%

%%%%%%%%%%%%%%%%%%%%%%%%%%%%%%%%%%%%%%%%%%%%%%%%%%%%%%%%%%%%%%%%%%%%%%%%%%%%
%%%                  Margins, Paragraphs, and Page                       %%%
%%%%%%%%%%%%%%%%%%%%%%%%%%%%%%%%%%%%%%%%%%%%%%%%%%%%%%%%%%%%%%%%%%%%%%%%%%%%
% LaTeX allows you to choose the margins for your document by defining     %
% them programmatically. Below are some default values that look nice.     %
%%%%%%%%%%%%%%%%%%%%%%%%%%%%%%%%%%%%%%%%%%%%%%%%%%%%%%%%%%%%%%%%%%%%%%%%%%%%
\oddsidemargin 0cm
\topmargin -2cm
\textwidth 16.5cm
\textheight 23.5cm

\setlength{\parindent}{0pt}
\setlength{\parskip}{5pt plus 1pt}

\pagestyle{fancyplain}
\lhead{\fancyplain{}{\textbf{HW\myhwnum}}}
\rhead{\fancyplain{}{\myname\\ \myandrew}}
\chead{\fancyplain{}{\mycourse}}
%%%%%%%%%%%%%%%%%%%%%%%%%%%%%%%%%%%%%%%%%%%%%%%%%%%%%%%%%%%%%%%%%%%%%%%%%%%%

%%%%%%%%%%%%%%%%%%%%%%%%%%%%%%%%%%%%%%%%%%%%%%%%%%%%%%%%%%%%%%%%%%%%%%%%%%%%
%%%                        HW Assignment Commands                        %%%
%%%%%%%%%%%%%%%%%%%%%%%%%%%%%%%%%%%%%%%%%%%%%%%%%%%%%%%%%%%%%%%%%%%%%%%%%%%%
% LaTeX allows you to create "commands", which can be reused over and      %
% over.                                                                    %
% In practice, these work a lot like methods/functions in programming.     %
%                                                                          %
% When typesetting homework, you will almost certainly need to separate    %
% questions and parts of questions.  So, we have defined a question        %
% command and a part command.                                              %
%%%%%%%%%%%%%%%%%%%%%%%%%%%%%%%%%%%%%%%%%%%%%%%%%%%%%%%%%%%%%%%%%%%%%%%%%%%%

\def\indented#1{\list{}{}\item[]}
\let\indented=\endlist

\newcounter{questionCounter}
\newcounter{partCounter}[questionCounter]
\newcommand{\question}[3][\arabic{questionCounter}]{
    \addtocounter{questionCounter}{1}
	\setcounter{partCounter}{0}
    \vspace{.25in} \hrule \vspace{0.5em}
        \noindent{\bf #1: #2}
    \vspace{0.8em} \hrule \vspace{.10in}
    #3
}
\renewcommand{\part}[2][\alph{partCounter}]{
    \addtocounter{partCounter}{1}
    \vspace{.10in}
    \begin{indented}
       {\bf (#1)} #2
    \end{indented}
}
\renewcommand{\maketitle}{
    \medskip
    \thispagestyle{plain}
    \begin{center}                  % Center the following lines
    {\Large \mycourse\;Assignment \myhwnum} \\
    \myname \\
    \myandrew \\
    Recitation \myrecitation\\
    \myduedate \\
    \end{center}
}
\newcommand{\graphic}[2][1.0]{
    \begin{center}
        \includegraphics[scale=#1]{#2}
    \end{center}
}
%%%%%%%%%%%%%%%%%%%%%%%%%%%%%%%%%%%%%%%%%%%%%%%%%%%%%%%%%%%%%%%%%%%%%%%%%%%%

%%%%%%%%%%%%%%%%%%%%%%%%%%%%%%%%%%%%%%%%%%%%%%%%%%%%%%%%%%%%%%%%%%%%%%%%%%%%
%%%          SKIP TO HERE IF YOU JUST WANT TO TYPE YOUR ANSWERS          %%%
%%%%%%%%%%%%%%%%%%%%%%%%%%%%%%%%%%%%%%%%%%%%%%%%%%%%%%%%%%%%%%%%%%%%%%%%%%%%

\begin{document}

%%%%%%%%%%%%%%%%%%%%%%%%%%%%%%%%%%%%%%%%%%%%%%%%%%%%%%%%%%%%%%%%%%%%%%%%%%%%
%%%                        Document Variables                            %%%
%%%%%%%%%%%%%%%%%%%%%%%%%%%%%%%%%%%%%%%%%%%%%%%%%%%%%%%%%%%%%%%%%%%%%%%%%%%%
% You should fill in the following information for each document you make. %
% \today is a LaTeX command that substitutes the current day's date; feel  %
% free to hardcode in a date if you'd like.                                %
%%%%%%%%%%%%%%%%%%%%%%%%%%%%%%%%%%%%%%%%%%%%%%%%%%%%%%%%%%%%%%%%%%%%%%%%%%%%
\newcommand{\mycourse}{15-150}
\newcommand{\myname}{Sri Raghavan}
\newcommand{\myandrew}{srikrish@andrew.cmu.edu}
\newcommand{\myhwnum}{9}
\newcommand{\myrecitation}{C}
\newcommand{\myduedate}{\today}
%%%%%%%%%%%%%%%%%%%%%%%%%%%%%%%%%%%%%%%%%%%%%%%%%%%%%%%%%%%%%%%%%%%%%%%%%%%%

\maketitle % This uses a command we defined previously to print the title

\newcommand{\code}[1]{\texttt{#1}}
\newcommand{\func}[2]{\textit{#1}(#2)}

%%%%%%%%%%%%%%%%%%%%%%%%%%%%%%%%%%%%%%%%%%%%%%%%%%%%%%%%%%%%%%%%%%%%%%%%%%%%
%%%                          Quick HowTo                                 %%%
%%%%%%%%%%%%%%%%%%%%%%%%%%%%%%%%%%%%%%%%%%%%%%%%%%%%%%%%%%%%%%%%%%%%%%%%%%%%
% If you want to write an answer to a question and let the template        %
% automatically number it for you:                                         %
%   \question{name}{                                                       %
%       answer                                                             %
%   }                                                                      %
%                                                                          %
% Optionally, you may specify the number of the question by doing:         %
%   \question[number]{name}{                                               %
%       answer                                                             %
%   }                                                                      %
%                                                                          %
% If the question has parts, you should do:                                %
%   \question{name}{                                                       %
%       \part{part a answer}                                               %
%       \part{part b answer}                                               %
%   }                                                                      %
%                                                                          %
% If you want to include an external graphic, you should do:               %
%   \graphic{path to graphic}                                              %
%                                                                          %
% Optionally, you may specify a decimal between 0 and 1 that represents    %
% the percentage that the image should be scaled by:                       %
%   \graphic[scale decimal]{path to graphic}                               %
%                                                                          %
% Syntax Reminders:                                                        %
%    Fractions: "a/b" is \frac{a}{b}                                       %
%    Binomial Coefficients: "a choose b" is \binom{a}{b}                   %
%    Subscripts: "a sub b" is a_{b}                                        %
%    Superscripts: "a to the b" is a^{b}                                   %
%    Greek Letters: "lower case alpha" is \alpha                           %
%                   "upper case alpha" is \Alpha                           %
%    Summations: "the summation from k=1 to n of k" is \sum_{k=1}^{n}{k}   %
%%%%%%%%%%%%%%%%%%%%%%%%%%%%%%%%%%%%%%%%%%%%%%%%%%%%%%%%%%%%%%%%%%%%%%%%%%%%

%% Example Question with Parts %%
\section*{4 The Blame Game}
    \subsection*{Task 4.1}
        As a preliminary goal, we'd like to identify the earliest point at which
        there exists an invariant violation (that is, we only know that the 
        invariant is violated at the point when we add point \code{p4} to our
        collection). We can print the same data about \code{cube3} (before the
        point is added), and we note that the invariant is still violated here.
        
        Tracing further back, we print data about \code{cube2} and note that the
        invariants hold at this point. Thus, we have identified that the invariant
        was breached at \code{cube3} - that is, by the adding of \code{p3} to
        the collection.
        
        We have two possibilities here. Either the point was constructed incorrectly,
        and thus the error is in \code{Cube.Box.Space.fromcoord
        (SeqUtil.seqFromTriple (2.0, 3.5, ~0.5))}, or the point was added incorrectly,
        pointing to \code{Cubes.addPoint (p3, cubes2) threshold} as the source
        of error.
        
        Note from section 4.1 that the structure \code{Cubes} is found in \code{cubes.sml}.
        Opening \code{cubes.sml}, we find that the structure \code{Box}
        is actually referencing \code{BoundingCube}. We note from section 4.1 that
        \code{BoundingCube} is implemented in \code{bcube.sml}. Looking in this
        file, we find that \code{Space} references \code{Space3D}, which section 4.1
        tells us is in \code{space3D.sml}. Thus, we look at this file. The function
        \code{fromcoord} calls \code{SeqUtil.tripleFromSeq} on a \code{real Seq.seq};
        we further note from above that we call \code{SeqUtil.seqFromTriple} and pass
        the result to this function. Thus, we'll check both of these functions.
        
        Looking at \code{sequtil.sml}, the source for \code{SeqUtil}, we find that
        both functions appear to be written correctly. Thus, we defer to our 
        second possibility, that the point was constructed correctly, but added
        incorrectly. Thus, we want to look at \code{Cubes.addPoint}, which is found
        in the structure \code{Cubes} in \code{cubes.sml}. On an unrelated note,
        there is a ridiculous amount of stuff going on in this function, and that
        means that it's gonna be tough to debug. At least we found the location.
        
        So we embark on taking apart this function to figure out how it works.
        First, the function checks to see if the point is already contained in
        our polytope sequence; if so, there's no need to expand or add a box.
        If not, we use \code{Seq.mapreduce} to find the polytope with center
        closest to the point to be added, and the distance from the center of this
        polytope to the point. We note that this code references code elsewhere
        at \code{Box.addPoint}. From section 4.1, we know that \code{Box} is
        located in \code{bcube.sml}.
        
        We note that \code{addPoint} here is just entirely broken. The code
        confuses the two points for each other; furthermore, some of the coordinates
        are mixed up. The relevant corrected code is below.
        
        \begin{verbatim}
        ...
          val (xleft, ylow, zfront) = SeqUtil.tripleFromSeq sfll
          val (xright, yup, zback) = SeqUtil.tripleFromSeq sbur
        in
          C((Space.fromcoord o SeqUtil.seqFromTriple) (Real.min (x, xleft),
                                                       Real.min (y, ylow),
                                                       Real.min (z, zfront)),
            (Space.fromcoord o SeqUtil.seqFromTriple) (Real.max (x, xright),
                                                       Real.max (y, yup),
                                                       Real.max (z, zback)))
        ...
        \end{verbatim}

        We substitute the correct code for the broken code, and we can
        run \code{boxclient.sml} to confirm that it's now correct. 
        
        Dishwasher? Yes.

    \subsection*{Task 4.2}
        We can trace back the execution through modules to find out exactly which
        module has incorrect code. The function we're calling is
        \code{Cubes.Box.center}. There is, thus, a function \code{center} located
        within the module \code{Box}, which is in turn referenced in the module
        \code{Cubes}.
        
        In section 4.1, we note that the structure \code{Cubes} is found in \code{cubes.sml}.
        Opening \code{cubes.sml}, we find that the structure \code{Box}
        is actually referencing \code{BoundingCube}. We note from section 4.1 that
        \code{BoundingCube} is implemented in \code{bcube.sml}.
        
        Opening \code{bcube.sml}, we find the function \code{center}, which
        uses \code{Space.midpoint fll bur} to calculate the center. This is,
        assuming \code{Space.midpoint} returns the correct result, a valid method
        for calculating the center of a polytope as defined here. However, because
        we don't return the correct result, \code{Space.midpoint} must continue
        the chain of error. We note that the structure \code{Space} in 
        \code{bcube.sml} is referencing the structure \code{Space3D}, which we
        can see (from section 4.1) is in \code{space3D.sml}.
        
        Sure enough, looking at \code{Space.midpoint}, we see that it calculates
        the midpoint using the code 
        \code{P((x1 + y1) / 2.0, (x2 + y2) / 2.0, (z1 + z2) / 2.0)}. However, 
        note that the midpoint of the $x$ and $y$ coordinates is being calculated
        incorrectly, because \code{y1} and \code{x2} are switched. Thus, to correct
        the bug, swap the places of \code{y1} and \code{x2}.
    
\section*{5 Serialization}
    \subsection*{Task 5.3}
        The problem with serialization of strings is that there's no obvious way
        to differentiate the SML data (meaning the actual contents of the string)
        from the way it's encoded into a string. Because we can have any character
        in a string (for example: parentheses) it's hard to know when to stop
        matching.
        
        There are a few ways to resolve this. One method involves recalling the
        fact that we can explode any \code{string} into a \code{char list}.
        We can serialize a string, then, by taking advantage of SerializeList.
        We can write SerializeChar, and then encode using SerializeList(SerializeChar).
        However, when reading back, we wouldn't know to decode the \code{char list}
        as a \code{string} rather than a literal \code{char list}. We could use
        a wrapper token, then, such as \code{(STR x)}, where \code{x} is 
        our serialized \code{char list}, to signal that the internal list is 
        actually representing a string.
        
        Another method might involve encoding the start and end of the string using
        special characters (such as the null character \code{$\backslash$0}),
        rather than parentheses. Of course, this is vulnerable to the fact that
        if we so desire, we can arbitrarily put null characters in strings; 
        in practice, however, this might be a useful (and more efficient) method
        than dealing with \code{char list}'s - users don't often input null characters.
        
        
\end{document}
